\documentclass[12pt, a4paper]{article}
\usepackage[utf8]{inputenc}
\usepackage[portuguese]{babel}
\usepackage{titlesec}
\usepackage{titling}
\usepackage{indentfirst}
\usepackage{graphicx}
\graphicspath{{./images/}}
\usepackage{wrapfig}
\usepackage{fancyhdr}
\usepackage{colortbl}
\usepackage{color}
\usepackage{framed}
\usepackage{enumitem}
\usepackage{amsmath}
\usepackage{lastpage}
\usepackage[hyphens]{url}
\usepackage{hyperref}
%\usepackage[brazilian,hyperpageref]{backref}
%\usepackage[num,overcite,abnt-emphasize=bf]{abntex2cite}
%%\usepackage[alf,abnt-emphasize=bf]{abntex2cite}
%%\citebrackets()
%\citebrackets[]

\hypersetup{
  colorlinks=true,
  linkcolor=black,
  filecolor=magenta,      
  urlcolor=blue,
  citecolor=black,
}

%% Definindo o Autor e o título
\newcommand{\prof}{Roberto Gil}
\newcommand{\materia}{Engenharia de Software}

\author{Ana Paula Merencia \\ Joana Pacheco Rolim \\ Milena Lucas dos Santos}
\title{Documento de Requisitos e Cronograma Macro}
\date{29 de junho de 2021}

%% zera a pagina
\fancyfoot[C]{}
%% linhas no inicio e fim da página
\renewcommand{\headrulewidth}{0.7pt}
\renewcommand{\footrulewidth}{0.5pt}

%% Definindo espaçamento
%\titlespacing{\section}{0pt}{*2}{*1.1}
%%\titlespacing{\subsection}{}{}{}

%%Definindo formato de títulos
%%\titleformat{comando}[formato, ex:wrap]{mudar fontes}{antes do separador}{separador}{depois do separador}[no fim do comando]
\titleformat{\section}
{\large}
{\thesection}
{.2cm}
{}[\titlerule]

\begin{document}
\begin{titlepage}
  \centering
  \thispagestyle{fancy}

  \begin{minipage}{0.4\textwidth}
    \begin{flushleft}
      \includegraphics[scale=0.6]{logoUnioeste.jpeg}\\[1.0 cm]
    \end{flushleft}
  \end{minipage}
  \begin{minipage}{0.5\textwidth}
    \begin{flushright}\large
      \textsc{\LARGE\textbf{UNIOESTE}}\\
      \vspace{1cm}
      Universidade Estadual\\do Oeste do Paraná
    \end{flushright}
  \end{minipage}
  %\rule{\textwidth}{.5pt}\\[2.0 cm]
  \vspace*{4.5 cm}

  {\huge\bfseries\thetitle}\\
  \rule{\linewidth}{0.2 mm}\\[1.5 cm]
  
  \vspace{2cm}
  \begin{minipage}[t]{0.4\textwidth}
    \begin{flushleft}\large
      \emph{Professor:}\\
      \prof
    \end{flushleft}
  \end{minipage}
  \begin{minipage}[t]{0.5\textwidth}

    \begin{flushright}\large
      \emph{Alunas:}\\
      \theauthor
    \end{flushright}

  \end{minipage}\\[2 cm]
  
  \vfill\thedate
  \end{titlepage}

  \pagestyle{fancy}
  \fancyfoot[L]{Matéria:~\materia}
  \fancyfoot[R]{Prof:~\prof}
  \fancyhead[L]{Alunas:~\theauthor}
  \fancyhead[R]{página:~\thepage/\pageref{LastPage}}

  \tableofcontents
  \section{Necessidades}
  \begin{itemize}
  	\item NC01 - Apresentar ao cliente um menu organizado de cada prato, mostrando os ingredientes de cada prato.
  	
  	
  	\item NC02 - Otimizar o tempo de escolha dos pratos
  	
  	\item NC03 -  Otimizar o tempo do pedido do cliente
  	
  	\item NC04 - Chamar atenção do cliente para o menu organizado por nacionalidade por meio de imagens dos pratos.
  	
  	\item NC05 - Disponibilizar um QRcode em cada mesa para os clientes terem acesso ao sistema para verem o cardápio e também  fazer o pedido, se for da preferência do cliente.
  	
  	\item NC06 - O cliente pode fazer o pedido estando no restaurante sem a necessidade do garçom.
  \end{itemize}
 
 
  \section{Requisitos Funcionais}
  \begin{itemize}
  	\item RF01 - Possibilitar um cardápio setorizado de acordo com as nacionalidades e a descrição dos ingredientes, apresentando imagens de cada prato.
  	
  	\item RF02 - Possibilitar ao cliente realizar o pedido pelo software no restaurante.
  	
  	\item RF03 - Possibilitar o cliente de realizar o pedido por delivery.
  	
  	\item RF04 - Possibilitar o cliente pedir o atendimento ao garçom.
  	
  	\item RF05 - O cliente tem acesso ao software por meio de um QR code no restaurante.
  	
  	\item RF06 - Para cada opção culinária deve ter entrada, prato principal e sobremesas.
  	
  	\item RF07 -  As opções culinárias dividem-se em nacionalidade culinária (brasileira e francesa), kids, vegetariana, bebidas e escolha do chef.
  	
  	\item RF08 - Possibilitar formas de pagamento quando delivery.
  	
  	\item RF09 - O cliente confirmando o pedido no software, o pedido é enviado para a cozinha por meio de um pdf ou e-mail.
  \end{itemize}

\section{Requisitos não Funcionais}
  \begin{itemize}
		\item RNF01 - O sistema deve enviar o pedido em um pdf para o WhatsApp.
		
		\item RNF02 - O sistema deverá funcionar em todos os navegadores de internet.
		
		\item RNF03 - O sistema deverá ter disponibilidade durante o funcionamento do restaurante.
		
		\item RNF04 - O sistema deverá se comunicar com o banco MySQL Server.
  \end{itemize}


\section{Requisitos Inversos}
\begin{itemize}
	\item RI01 - O cliente não poderá fazer o pagamento pelo software estando no restaurante.
	\item RI02 - O software não realizará pagamento pelo cartão de débito.
	
	
\end{itemize}


\end{document}
